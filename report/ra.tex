\section{RANGE ANALYSIS}
In this section, we discuss range analysis. We use mem2reg pass for range analysis.

\subsection{Lattice and Flow Function}
We define complete lattice:
\begin{equation}
Z = \mathbb{Z} \cup \{-\infty, +\infty\}
\end{equation}
for arithmetic operations. 
From $Z$, we define product lattice
\begin{equation}
Z^2 = \{\emptyset \} \cup \{[z_1, z_2] | z_1, z_2 \in Z, z_1 \le z_2, -\infty < z_2 \}
\end{equation}

$\sqsubseteq$ is ordered by subset relation. 

Range intersection $\sqcap$:
\begin{equation}
[a_1, a_2] \sqcap [b_1, b_2] =
\begin{cases}
[max(a_1, b_1), min(a_2, b_2)], b_1 \in [a_1, a_2] \ or \  a_1 \in [b_1, b_2] \\
\emptyset , \ else
\end{cases}
\end{equation}

Range union $\sqcup$:
\begin{equation}
[a_1, a_2] \sqcup [b_1, b_2] = [min(a_1, b_1), max(a_2, b_2)]
\end{equation}

$V$ is the set of constraint variables.

\begin{equation}
I: V -> Z^2
\end{equation}
mapps from constraint variables to intervals in $Z^2$. Our lattices and flow functions are given above.

\subsection{Implementation}

\subsection{Benchmark}
We provide three brenchmarks for range analysis, including RAbasic, RAbranch1 and RAbranch2. Source C++ code (RAbasic) and LLVM IR are given below:

\lstinputlisting{../extra/benchmarks/RAbasic/RAbasic.cpp}



\subsection{Discussion}