\section{AVALIABLE EXPRESSIONS ANALYSIS}
In this section, we present available expressions analysis, and the Common Sub-Expression Elimination optimization that uses the analysis. 

\subsection{Assumptions}
We run this analysis on the basis of mem2reg pass, SSA form makes this task much simpler. It is a must analysis.

\subsection{Lattice and Flow Function Definition}

\begin{align} 
	\begin{split}
		(D, \top, \bot, \sqcup, \sqcap, \sqsubseteq) = (2^{x \to y | x \in Vars \wedge y \in Exprs}, \emptyset, x \to y | x \in Vars \wedge y \in Exprs, \cap, \cup, \subseteq)
	\end{split}
\end{align}

In SSA, we don't care if we will overwrite an existing variable. Each assignment will produce a new variable in SSA.
\begin{align} 
	\begin{split}
		F_{X := Y op Z}(in) = in  \cup \{ X \to Y op Z \}
	\end{split}
\end{align}

In LLVM SSA, there can't be three branches in a phi node.
\begin{align} 
	\begin{split}
		F_{X := \Phi(Y_1, Y_2)}(in_1, in_2) = in_1 \cap in_2  \cup \{x \to y | x \to y \in in_1 \wedge x \to y \in in_2\}
	\end{split}
\end{align}

We have defined our lattice and flow functions above.

\subsection{Implementation}
We use a map to represent information of a lattice node. The key of map is \texttt{llvm::Value*}, and the value of map is \texttt{llvm::Instruction*}. We map a variable to the instruction where it is first defined. \\

To check if an instruction is already defined is very simple in LLVM. For example, if we have following codes :
\begin{verbatim}
L1 : %add = add nsw i32 1, 2
L2 : %add1 = add nsw i32 1, 2
\end{verbatim}
Then by call API \emph{isIdenticalToWhenDefined}, we can check L1 and L2 are the same. As a result, we map \%add and \%add2 to the same instruction. \\

When join, we perform intersection on left side input and right side input.

\subsection{Common Sub-Expression Elimination Optimization}

In this optimization, our pass searches the LLVM IR for instances of identical expressions, where all the expressions evaluate to the same value, and uses the available expressions analysis to determine whether the expressions should be replaced. 

Let's review the above code. Generally speaking, we want to transform out code like this:
\begin{verbatim}
L1 : %add = add nsw i32 1, 2
L2 : %add1 = %add
\end{verbatim}
However, when we apply \texttt{mem2reg} pass, there is no single assignment instruction and we can't transform code like this. So I apply a general optimization using our available expressions analysis. Looking at the following code:

\begin{verbatim}
L1 : %add = add nsw i32 1, 2
L2 : %add1 = add nsw i32 1, 2
L3 : ret i32 %add1
\end{verbatim}

We know that \%add1 is pointing to L1, so we use \%add to replace \%add1 in ret instruction, then L2 is dead code, we just simply remove it. By this way, we transform original code like this:
\begin{verbatim}
L1 : %add = add nsw i32 1, 2
L3 : ret i32 %add
\end{verbatim}

It is a useful optimization.

\subsection{Benchmark}
We provide 4 benchmarks for available expression analysis, including CSEbasictest, CSEbranch, CSEbranch2, and CSEphi.

Here I show an interesting test case named CSEbranch2 :
	
\begin{verbatim}
	int f1 = 1;
	int f2 = 2;
	int a = f1 + f2;
	if(a > 0) {
	    int b = f1 + f2;
	} else {
	    int c = f2 + f1;
	}
	return 1;
\end{verbatim}

In this case, c = f2 + f1 is the common expression of a = f1 + f2. But the API isIdenticalToWhenDefined can't recognize this. Because this method checks operands one by one, so it return false. But it is not the case because they are definitely same. So When dealing with add instruction, we add one more check : we swap the operands and call isIdenticalToWhenDefined one more time to ensure we can handle this case.

\subsection{Discussion}
Consider the following code
\begin{verbatim}
entry:
    %cmp = icmp sgt i32 3, 0
    br i1 %cmp, label %if.then, label %if.else

if.then:                                          ; preds = %entry
    %add = add nsw i32 1, 2
    br label %if.end

if.else:                                          ; preds = %entry
    %add1 = add nsw i32 1, 2
    br label %if.end

if.end:                                           ; preds = %if.else, %if.then
    %add2 = add nsw i32 1, 2
    ret i32 %add2
\end{verbatim}
For \%add2, we know add new i32 1, 2 is an available expression, and we could transform code like this:
\begin{verbatim}
if.end:
    %add2 = phi i32 [ %add, %if.then ], [ %add1, %if.else ]
\end{verbatim}
But I didn't apply this transformation. If you have many branches, then this method will create many basic blocks and phi nodes, which will bring unnecessary transformations. Because calculating 1+2 is so cheap, we didn't do optimizations for it.