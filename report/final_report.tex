%%%%%%%%%%%%%%%%%%%%%%%%%%%%%%%%%%%%%%%%%%%%%%%%%%%%%%%%%%%%%%%%%%%%%%
% LaTeX Example: Project Report
%
% Source: http://www.howtotex.com
%
% Feel free to distribute this example, but please keep the referral
% to howtotex.com
% Date: March 2011 
% 
%%%%%%%%%%%%%%%%%%%%%%%%%%%%%%%%%%%%%%%%%%%%%%%%%%%%%%%%%%%%%%%%%%%%%%
% How to use writeLaTeX: 
%
% You edit the source code here on the left, and the preview on the
% right shows you the result within a few seconds.
%
% Bookmark this page and share the URL with your co-authors. They can
% edit at the same time!
%
% You can upload figures, bibliographies, custom classes and
% styles using the files menu.
%
% If you're new to LaTeX, the wikibook is a great place to start:
% http://en.wikibooks.org/wiki/LaTeX
%
%%%%%%%%%%%%%%%%%%%%%%%%%%%%%%%%%%%%%%%%%%%%%%%%%%%%%%%%%%%%%%%%%%%%%%
% Edit the title below to update the display in My Documents
%\title{Project Report}
%
%%% Preamble
\documentclass[paper=letter, fontsize=11pt]{scrartcl}
\usepackage[margin=1in]{geometry}
\usepackage[T1]{fontenc}
\usepackage{fourier}

\usepackage[english]{babel}															% English language/hyphenation
\usepackage[protrusion=true,expansion=true]{microtype}	
\usepackage{amsmath,amsfonts,amsthm} % Math packages
\usepackage[pdftex]{graphicx}	
\usepackage{url}
\usepackage{listings}

%%% Custom sectioning
\usepackage{sectsty}
\allsectionsfont{\normalfont\scshape}


%%% Custom headers/footers (fancyhdr package)
\usepackage{fancyhdr}
\pagestyle{fancyplain}
\fancyhead{}											% No page header
\fancyfoot[L]{}											% Empty 
\fancyfoot[C]{}											% Empty
\fancyfoot[R]{\thepage}									% Pagenumbering
\renewcommand{\headrulewidth}{0pt}			% Remove header underlines
\renewcommand{\footrulewidth}{0pt}				% Remove footer underlines
\setlength{\headheight}{13.6pt}


%%% Equation and float numbering
\numberwithin{equation}{section}		% Equationnumbering: section.eq#
\numberwithin{figure}{section}			% Figurenumbering: section.fig#
\numberwithin{table}{section}				% Tablenumbering: section.tab#


%%% Maketitle metadata
\newcommand{\horrule}[1]{\rule{\linewidth}{#1}} 	% Horizontal rule

\title{
		\vspace{-0.5in} 	
		\usefont{OT1}{bch}{b}{n}
		\normalfont \normalsize \textsc{University of California, San Diego \\
								Department of Computer Science and Engineering } \\ [25pt]
		\horrule{0.5pt} \\[0.4cm]
		\huge CSE 231 Project Report\\
		\horrule{2pt} \\[0.5cm]
}
\author{
		\normalfont 						\normalsize
        				Zhenchao Gan\\[-3pt]		\normalsize
				Antonella Wilby\\[-3pt]		\normalsize
        				Tao Li\\[-3pt]				\normalsize
				\{zhgan, awilby, taoli\} @eng.ucsd.edu \\ \\
       		 \today
}
\date{}


%%% Begin document
\begin{document}
\maketitle
\section{OVERVIEW}

In this project we implemented an intra-procedural dataflow analysis and optimization framework in LLVM.  Our framework first builds a control flow graph using LLVM's intermediate representation (IR) code. The framework then runs the worklist algorithm on the IR code, applying flow functions until the analysis reaches a fixed point. This framework takes a lattice bottom node and corresponding flow function to start with, and iterates repeatedly until the entire dataflow analysis is built. 

We implemented four analyses using our framework: (1) constant propagation, (2) available expressions analysis, (3) range analysis and (4) intra-procedural pointer analysis. The first three analyses are based on the output of LLVM's mem2reg pass. The mem2reg pass promotes memory references to be register references. It promotes alloca instructions which only have loads and stores as uses. This pass is very useful such that we will not be disturbed by complicated memory problem. However, this pass has helped us finish pointer analysis, so before applying intra-procedure pointer analysis we have to close this pass.

Using these analyses, we implemented a series of optimizations. We used constant propagation to implement constant folding and branch folding optimizations. We used the available expression analysis to apply common subexpression elimination and dead code elimination.  Based on range analysis, we implemented a bug finder to warn the programmer if can’t show array access in bounds. Because these analysis are based on mem2reg pass, our own intra-procedure analysis is just to present information we gathered.


\section{ANALYSIS FRAMEWORK}

In this section, we present our data analysis framework interface design and introduce important APIs that are used in the worklist algorithm.

\subsection{Interface Design}

We will discuss two base classes:  \texttt{LatticeNode}, which represents a node in a lattice,  and \texttt{FlowFunction}, which is used to define flow functions for each separate analysis. \\

\texttt{LatticeNode.h} is the abstract base class of the four concrete lattice nodes. The common attributes of LatticeNode include:
\begin{enumerate}
\item \texttt{istop}:  represents whether this lattice node is in top.
\item \texttt{isbottom}: represents whether this lattice node is in bottom.
\item \texttt{kind}: represents the kind of this lattice node (e.g. Constant Prop lattice, Available Expressions lattice, etc.).
\end{enumerate}

Additionally, it provides the following common APIs:
\begin{enumerate}
\item \texttt{join}: a function that provides a way to join two lattice nodes.
\item \texttt{PrintInfo}: a function that dumps information about this lattice node.
\end{enumerate}

We don't need \texttt{meet} method, because the lattice is built from bottom to top, only upward. It is impossible to go downwards. \\

\texttt{FlowFunction.h} is the base class of the four concrete flow functions. We overrided the () operator, making it much more like a function. () operator defines what we will get after going through an instruction when given input. \\

We implement these interfaces for each of our four analyses. Each analysis is implemented in a subclass which inherits the base class in order to implement the interface, as follows:
\begin{enumerate}
\item Constant Propagation is implemented in \texttt{CPLatticeNode} and \texttt{CPFlowFunction}.
\item Available Expressions analysis is implemented in \texttt{AELatticeNode} and \texttt{AEFlowFunction}.
\item Range Analysis is implemented in \texttt{RALatticeNode} and \texttt{RAFlowFunction}.
\item Intra-procedural Pointer Analysis is implemented in \texttt{PALatticeNode} and \texttt{PAFlowFunction}.
\end{enumerate}

All of these analyses are implemented as subclasses which inherit the base classes of \texttt{LatticeNode} and \texttt{FlowFunction} in order to implement the interface.

\subsection{Worklist}
Our worklist algorithm takes three parameters, a procedure \texttt{F}, a starting lattice node initialized at bottom and a flow function. Then we will run a forward optimistic iterative dataflow analysis that computes a piece of information at the beginning of each instruction.

The control flow graph in LLVM is based on basic blocks and in LLVM there is no explicit representation of edges. Because of this, the best way to store information while doing your fixed point computation is to store one piece of information for the beginning of each basic block. To represent the edges, we use pre\_block--post\_block pair as key and lattice node as value to store information. It is easy with LLVM to build the pre\_block and post\_block relationship. Then we apply block-level flow function on the control flow graph.

When we reached a fixed point, we get the information at the beginning of each basic block. Next we should propagate the information to each instruction. So, in each instruction within a block, we run instruction-level flow function and finally get the information at the beginning of each instruction.

\subsection{Optimization Passes}

For each analysis, we create a corresponding pass to make use of the information produced by the analysis. The optimizations are as follows:
\begin{enumerate}
\item Constant Propagation is used for the Constant Folding and Branch Folding optimizations, implemented in \texttt{CFpass.cpp} and \texttt{BFpass.cpp}, respectively.
\item Available Expressions Analysis: used for Common Sub-expression Elimination optimization, implemented in \texttt{CSEpass.cpp}.
\item Range Analysis: used for a bug-finder to warn a programmer if an array access can't be shown within bounds, implemented in \texttt{RApass.cpp}.
\item Pointer Analysis: combines all other analyses for an improved optimization, implemented in \texttt{PApass.cpp}.	
\end{enumerate}


\section{CONSTANT PROPAGATION}

In this section, we present our Constant Propagation analysis, and the two optimizations implemented using this analysis: Constant Folding and Branch Folding.

\subsection{Assumptions}

We run this analysis on the output of \texttt{mem2reg}. 

\subsection{Lattice and Flow Function Definition}

We define the lattice for constant propagation analysis as follows:

\begin{align} 
	\begin{split}
		(D, \top, \bot, \sqcup, \sqcap, \sqsubseteq) = (2^{\{x \to N | x \in Vars \wedge N \in \mathbb{Z}\}}, \emptyset, \{x \to N | x \in Vars \wedge N \in \mathbb{Z}\}, \cap, \cup, \subseteq)
	\end{split}
\end{align}

The domain is a mapping from variables to constant integer values. $\top$ is the most conservative lattice element, where every variable is non-constant. $\bot$ is the least conservative, where every variable can be a constant. For example, $\{a \rightarrow 8\}$ implies $a$ is a constant with value 8.

The flow functions for assignment, binary operators, and PHI are defined as follows:

\begin{align} 
	\begin{split}
		F_{X := N}(in) = in  - \{ X \to *\} \cup \{ X \to N \}
	\end{split}
\end{align}

\begin{align} 
	\begin{split}
		F_{X := Y op Z}(in) = in  - \{ X \to *\} \cup \{ X \to N | (Y \to N_1) \in in \wedge \\
		(Z \to N_2) \in in \wedge \\
		N = N_1 op N_2\}
	\end{split}
\end{align}

\begin{align} 
	\begin{split}
		F_{X := \Phi(Y_1, Y_2)}(in_1, in_2) = in_1 \cap in_2  \cup \\
		\{x \to y | x \to y \in in_1 \wedge \\
		 x \to y \in in_2\}
	\end{split}
\end{align}

\subsection{Implementation}

In our implementation we represent lattice nodes with a map from \texttt{llvm::Value*} to \texttt{llvm::Constant*}. For simplicity of implementation, we only handle integer constants, but the implementation can be extended to handle other data types. \\

To check if 

\subsection{Constant Folding Optimization}

We use our constant propagation analysis to implement a constant folding optimization pass. The constant folding pass finds constant expressions, like $1+2$, and folds them to eliminate the add instruction. 

\subsection{Branch Folding Optimization}

We use our constant propagation analysis to implement a branch folding optimization pass. In this optimization, if the conditional expression causing a branch can be evaluated at compile time (i.e., the value of the comparison is known to be true or false). We can use that result of constant propagation to eliminate branches that are never taken. 

\subsection{Benchmark}

\subsection{Discussion}
\section{AVALIABLE EXPRESSION ANALYSIS}
In this section, we present available expression analysis.

\subsection{Assumptions}
We run this analysis on the basis of mem2reg pass, SSA form makes this task much simpler. It is a must analysis.

\subsection{Lattice and Flow Function}

\begin{align} 
	\begin{split}
		(D, \top, \bot, \sqcup, \sqcap, \sqsubseteq) = (2^{x \to y | x \in Vars \wedge y \in Exprs}, \emptyset, x \to y | x \in Vars \wedge y \in Exprs, \cap, \cup, \subseteq)
	\end{split}
\end{align}

In SSA, we don't care if we will overwrite an existing variable. Each assignment will produce a new variable in SSA.
\begin{align} 
	\begin{split}
		F_{X := Y op Z}(in) = in  \cup \{ X \to Y op Z \}
	\end{split}
\end{align}

In LLVM SSA, there can't be three branches in a phi node.
\begin{align} 
	\begin{split}
		F_{X := \Phi(Y_1, Y_2)}(in_1, in_2) = in_1 \cap in_2  \cup \{x \to y | x \to y \in in_1 \wedge x \to y \in in_2\}
	\end{split}
\end{align}

We have defined our lattice and flow functions above.

\subsection{Implementation}


\subsection{Optimization}


\subsection{Benchmark}
\section{RANGE ANALYSIS}
\section{INTRA-PROCEDURE POINTER ANALYSIS}

\subsection{Assumptions}


\subsection{Lattice and Flow Function}


\subsection{Implementation}


\subsection{Benchmark}


\section{RANGE ANALYSIS}
%%% End document
\end{document}