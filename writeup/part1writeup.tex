\documentclass[12pt]{article}

\usepackage[letterpaper, portrait, margin=1in]{geometry}

\begin{document}

\title{CSE 231 Project Report: LLVM Project}
\author{Zhenchao Gan, Tao Li, Antonella Wilby}
\date{November 3, 2015}
\maketitle

\section*{Overview}

In this project, we implemented LLVM passes for collecting static instruction counts, dynamic instruction counts, and profiling branch biases. Each LLVM pass takes in a program in its Intermediate Representation (IR) and uses that information to profile the program. In section one, we describe a pass that counts the number of instructions in the IR, which is the static instruction count (IC). In section two, we describe a pass that counts the number of instructions executed at runtime, or the dynamic IC. In section three, we describe a pass that analyzes the branch bias for each function in a program, i.e. the taken/not-taken ratio of the branch instructions in the program. We used each pass to profile four benchmarks: {\tt welcome.cpp}, {\tt compression.c}, {\tt gcd.cpp}, and {\tt hadamard\_test.c}.

\section{Collecting Static Instruction Counts}

We implemented an LLVM pass that takes in the Intermediate Representation of a program, and counts the number of occurrences of each separate instruction, then prints the results after the program has been analyzed.

\subsection{Algorithm and Implementation}

We defined our pass {\tt csi} as a struct inheriting from {\tt ModulePass}. We first create a C++ map to store the instructions encountered while profiling the program and the associated instruction counts. The algorithm first iterates through a given Module (e.g., {\tt hadamard.bc}). Within the Module, the algorithm iterates through every function, and within the function the algorithm iterates through each basic block. As it iterates through the basic block, the algorithm gets instructions using the function {\tt Instruction::getOpcodeName()} and increments the associated counter within the map for each instruction in the basic block. After the entire IR has been analyzed, the algorithm prints the contents of the map to {\tt stdout}.

\subsection{Benchmark}

\begin{figure*}[!t]
\begin{center}
\begin{tabular} { |c|c| }
\hline
	add & 526 \\
	br & 3143 \\
	call & 473 \\
	getelementptr & 2806 \\
	load & 6989 \\
	mul & 37 \\
	store & 3044 \\
	sub & 272 \\
	TOTAL & 22167 \\
\hline
\end{tabular}
\caption{Selected static instruction counts from the {\tt compression} benchmark}
\label{STATIC}
\end{center}
\end{figure*}

We discuss the results of our {\tt csi} pass on the {\tt compression} benchmark. Figure \ref{STATIC} shows static instruction counts for a select set of common instructions, with some lesser-used instructions omitted for brevity. Please refer to the logs for complete statistics.
\\\\
The total static instruction count for the program is 22167 instructions. As would be expected, the most common instructions include {\tt br} instructions, {\tt load} instructions, and {\tt store} instructions. 


\section{Collecting Dynamic Instruction Counts}

We implemented an LLVM pass for computing the dynamic instruction count of a program, or the number of executed instructions. Our pass instruments the program by inserting a call during execution to a function in our instrumentation module that counts the number of times each instruction executes. 

\subsection{Algorithm and Implementation}

The instrumentation module contains a map for collecting instruction counts, a function called {\tt collectRuntimeInfo()}, and a function called {\tt dumpInfo()}. The {\tt collectRuntimeInfo()} function takes in an opcode name and a counter for that opcode, and uses the counter to update the counter for that opcode stored inside the map. The {\tt dumpInfo()} function is called at the very end of the pass execution, and prints the contents of the instruction counter map to {\tt stdout}. 
\\\\
As in our first pass, this pass iterates through each module, function, and basic block in the program. For each basic block, the program creates a C++ map to store the instructions executed during execution of that block. If the program encounters any of the opcodes {\tt jump}, {\tt ret}, or {\tt br}, it means the end of the basic block has been reached, so we insert a call to our instrumentation module. We use the function {\tt Module::getOrInsertFunction()} and a call to {\tt IRBuilder::CreateCall()}  to insert a call to the instrumentation module's {\tt collectRuntimeInfo()} function in order to add the instruction counts gathered during this basic block to the overall instruction counter map.    In the pass, we call this instrumentation module function with the mangled function name {\tt \_Z18collectRuntimeInfoPci}. 
\\\\
After the entire program has executed, we use these same functions to insert a call to the {\tt dumpInfo()} function in the instrumentation module. To determine if it is the main function that is executing, we to see if the instruction is a {\tt return} function and if its parent function name is {\tt main}. If so, we use the {\tt Module::getOrInsertFunction()} and {\tt IRBuilder::CreateCall()} functions to insert a call to our {\tt dumpInfo()} function using the mangled function name {\tt \_Z8dumpInfov}. 


\subsection{Benchmark}

\begin{figure*}[!t]
\begin{center}
\begin{tabular} { |c|c| }
\hline
	add & 14606 \\
	br & 28848 \\
	call & 140 \\
	getelementptr & 33676 \\
	load & 86528 \\
	mul & 29 \\
	store & 29634 \\
	sub & 1780 \\
	TOTAL & 241847 \\
\hline
\end{tabular}
\caption{Selected dynamic instruction counts from the {\tt compression} benchmark}
\label{DYNAMIC}
\end{center}
\end{figure*}

We discuss the results of our {\tt cdi} pass on the {\tt compression} benchmark. Figure \ref{DYNAMIC} shows dynamic instruction counts for the same set of common instructions as in Section 1, with lesser-used instructions omitted. The total dynamic instruction count for the program is 241847 executed instructions. As expected, {\tt branch}, {\tt load}, and {\tt store} instructions continue to make up the majority of executed instructions, but the {\tt add} and {\tt getelementptr} instructions are also executed very frequently. The number of {\tt call} and {\tt mul} instructions decreased, suggesting that there are branches in the program which are never taken. 

\section{Profiling Branch Bias}

\subsection{Algorithm and Implementation}

\subsection{Benchmark}

We discuss the results of our {\tt pbb} pass on the {\tt compression} benchmark. 

\begin{figure*}[!t]
\begin{center}
\begin{tabular} { |c|c|c|c| }
\hline
	Function & Bias & Taken & Total \\
	main & 0.222222 & 2  & 9 \\
	mz\_adler32 & 0.955882 &  65  & 68 \\
	mz\_compress2 & 0.000000  & 0 & 3 \\
	mz\_deflate & 0.500000  & 5 & 10 \\ 
	mz\_deflateBound & 1.000000 & 1 & 1\\
	mz\_deflateEnd & 1.000000 & 2  &  2 \\
	mz\_deflateInit2  & 0.222222 & 2  & 9 \\
	mz\_inflate & 0.538462 & 7 &13 \\
	mz\_inflateEnd & 1.000000 & 2 & 2 \\
	mz\_inflateInit2 & 0.400000 & 2 & 5 \\
	mz\_uncompress & 0.000000 & 0 & 3 \\
	tdefl\_calculate\_minimum\_redundancy & 0.650246 & 264 & 406 \\
	tdefl\_compress  & 0.541667 & 13  & 24 \\
	tdefl\_compress\_block & 0.000000 & 0  & 1 \\
	tdefl\_compress\_lz\_codes & 0.704082 & 207& 294 \\
	tdefl\_compress\_normal & 0.611734 & 2127 & 3477 \\
	tdefl\_create\_comp\_flags\_from\_zip\_params & 0.285714 & 2 & 7 \\
	tdefl\_flush\_block & 0.666667 & 38 & 57 \\
	tdefl\_flush\_output\_buffer & 0.750000 & 3 & 4 \\
	tdefl\_huffman\_enforce\_max\_code\_size & 0.888889 & 96 &108 \\
	tdefl\_init & 0.000000 & 0 & 1 \\
	tdefl\_optimize\_huffman\_table & 0.756056 & 1342 & 1775 \\
	tdefl\_radix\_sort\_syms  & 0.979381 & 855  & 873 \\
	tdefl\_start\_dynamic\_block &  0.409742 & 715 &  1745 \\
	tinfl\_decompress  & 0.863445 & 4521 & 5236 \\
\hline
\end{tabular}
\caption{Branch biases from the {\tt compression} benchmark}
\label{BRANCH}
\end{center}
\end{figure*}

\end{document}